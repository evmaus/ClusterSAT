\documentclass[10pt]{article}

\title{CSE 507 Proposal}
\author{Everett Maus}

\begin{document}

\maketitle

\begin{abstract}
  We propose developing a SAT solver which allows clustering for faster solving, in the 
  vein of ManySAT, but communicating over network protocols to allow for scaling beyond
  a single computer.
\end{abstract}

\section{Background}
SAT Solvers are a key part of the program analysis toolbox.  However, they current research on
  them has focused largely on single machine SAT solving--even the parallel track in the SAT 
  Competition 2017 was on a single machine with multiple cores. \footnote{https://baldur.iti.kit.edu/sat-competition-2017/index.php?cat=tracks}

  Historically, a common model for speeding up SAT solving is the approach proposed in ManySAT,
  where many instances of a SAT Solver are run with slightly different parameters.  However,
  a cursory research examination makes it appear that these are really only run on single machines.

  In the meantime, the rise of commodity computing in the cloud has made lower performance, networked
  computation readily available.

\section{Precise Description}

  We propose two deliverables.  The first is "TribbleSAT", a CDCL-based SAT Solver, 
  written in C++, designed with pluggability of solving strategies in mind.  We don't anticipate
  this will be higher performance than a Glucose-based solver--instead, it will simply serve to
  prove out the value of the second deliverable.  Because of the focus on pluggability of different
  strategies, it should be possible to use "TribbleSAT" to quickly prototype new variable
  selection strategies, learned-clause compacting strategies, and similar tweaks on the classical
  algorithm, and test the behavior of those using the second deliverable, "ClusterSAT".

  The second deliverable is a pluggable SAT Solving layer named "ClusterSAT" that distributes
  TribbleSAT instances across one or more computers, communicating over a network protocol.  It will
  be designed to be largely SAT-solver independent, so that higher performance solvers could be 
  plugged into it in the future, but initially will only support "TribbleSAT" as the first solver.
  We propose using a standard leader/follower model for distributing this computation, 
  where SAT problems are submitted to a 'coordinator' (leader) node, which then passes them to one or more 
  'solver' (follower) nodes, each of which consists of a single solver instance.  When a single solver node finds
  a solution to the SAT problem (or all of them time out), the 'coordinator' will inform all of the shared clients.
  Time permitting, learned clause sharing will be enabled between 

   In terms of implementation, "ClusterSAT" will use the gRPC framework, and 
  communicate using protocol buffers--this allows each solver node (follower node) to be written in whatever
  language seems most appropriate for that solver.

\section{Related Literature}
  GridSAT:
  PSATO:
  ManySAT:

\section{Implementation Plan}



\end{document}