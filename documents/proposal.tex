\documentclass[10pt]{article}

\title{CSE 507 Proposal}
\author{Everett Maus}

\begin{document}

\maketitle

\begin{abstract}
  I propose developing a SAT solver which allows clustering for faster solving, in the 
  vein of ManySAT, but communicating over network protocols to allow for scaling beyond
  a single computer.
\end{abstract}

\section{Background}
SAT Solvers are a key part of the program analysis toolbox.  However, they current research on
  them has focused largely on single machine SAT solving--even the parallel track in the SAT 
  Competition 2017 was on a single machine with multiple cores. \footnote{https://baldur.iti.kit.edu/sat-competition-2017/index.php?cat=tracks}

  Historically, a common model for speeding up SAT solving is the approach proposed in ManySAT,
  where many instances of a SAT Solver are run with slightly different parameters.  However,
  a cursory research examination makes it appear that these are really only run on single machines.

  In the meantime, the rise of commodity computing in the cloud has made lower performance, networked
  computation readily available.

\section{Precise Description}

  I propose two deliverables.  The first is "TribbleSAT", a CDCL-based SAT Solver, 
  written in C++, designed with pluggability of solving strategies in mind.  I don't anticipate
  this will be higher performance than a Glucose-based solver--instead, it will simply serve to
  prove out the value of the second deliverable.  Because of the focus on pluggability of different
  strategies, it should be possible to use "TribbleSAT" to quickly prototype new variable
  selection strategies, learned-clause compacting strategies, and similar tweaks on the classical
  algorithm, and test the behavior of those using the second deliverable, "ClusterSAT".

  The second deliverable is a distributed SAT Solving layer named "ClusterSAT" which allows distributing
  TribbleSAT instances across one or more computers, communicating over a standard network protocol.  It will
  be designed to be largely SAT-solver independent, so that higher performance solvers could be 
  plugged into it in the future, but initially will only support "TribbleSAT" as the first solver.
  I propose using a standard leader/follower model for distributing this computation, 
  where SAT problems are submitted to a 'coordinator' (leader) node, which then passes them to one or more 
  'solver' (follower) nodes, each of which consists of a single solver instance.  When a single solver node finds
  a solution to the SAT problem (or all of them time out), the 'coordinator' will inform all of the shared clients.
  Time permitting, learned clause sharing will be enabled between follower nodes. In terms of implementation, 
  "ClusterSAT" will use the gRPC framework, and communicate using protocol buffers--this allows each solver node 
  (follower node) to be written in whatever language seems most appropriate for that solver.  The coordinator node 
  and TribbleSAT solver node implementations will be in C++.

\section{Related Literature}
  The most immediately relevant literature is GrADSAT (Chrabakh et al.), which attempted to distribute SAT solving
  across a computational grid.  However, GrADSAT approaches the problem of distributing SAT by splitting the SAT problem
  into sub-problems, an approach that has fallen out of popularity since the success of ManySAT (Hamadi et al.).

  ManySAT (Hamadi et al.) proposes the solving distributing SAT by taking advantage of the fact that the performance of
  modern SAT solvers is very dependent on their initial implementation, and parallelizes SAT by simply running multiple,
  carefully chosen solver implementations.  Our proposal is to use a similar model, but to distribute our computation across
  multiple machines to achieve better scalability.

\section{Implementation Plan}
  By February 7th, I intend to have a basic prototype working.  This will consist of a CDCL based SAT solver (TribbleSAT),
  a prototype ClusterSAT fabric, and a way to send a coordinator a SAT problem, have it farm that problem out to multiple 
  copies of TribbleSAT over the network, and aggregate the results from the solver nodes.

  By the end of February, I'll implement two-watched-literals instead of linear scanning for BCP, clause sharing across 
  TribbleSAT nodes and implement more restart/compacting/variable selection/etc. strategies in TribbleSAT.

  In the last few weeks of class, I'll be benchmarking the solver against single threaded TribbleSAT and Glucose on various
  problem set sizes, and, budget permitting (i.e. if I can use free cloud services for it), demonstrating it running on a
  cloud provider.


\end{document}